
\section{Introduction}


General intro predictive markets.


Prediction market is generally implemented as a wager (or contract) that 
pays off if a particular outcome, taking a particular value $y$, occurs.
Assuming that both the efficient markets hypothesis holds, 
and that the market acts as a risk-neutral representative trader, the price 
of the contract will be the best estimate of various parameters tied 
to the probability of that outcome \cite{wolfers2012prediction}.

\begin{table}[h]
    \centering
    \caption{Contract Types: Estimating Uncertain Quantities or Probabilities}
    \label{tab:contract_types}
    \begin{tabular}{lp{4cm}p{4cm}p{4cm}}
        \toprule
        \textbf{Contract} & \textbf{Example}                                                                 & \textbf{Details}                                                                                           & \textbf{Reveals market expectation of...} \\
        \midrule
        Winner-takes-all  & Outcome $y$: Level of initial unemployment claims (in thousands).                & Contract costs $p$. Pays \$1 if and only if event $y$ occurs. Bid according to the value of $p$.           & Probability that outcome $y$ occurs.      \\
        \midrule
        Index             & Contract pays \$1 for every 1,000 initial unemployment claims.                   & Contract pays \$ $y$.                                                                                      & Mean value of outcome $y$: $E[y]$.        \\
        \midrule
        Spread            & Contract pays even money if initial unemployment claims are greater than $y^{*}$ & Contract cost \$ 1. Pays \$2 if $y \ge y^{*}$. Pays \$ 0 otherwise. Bid according to the value of $y^{*}$. & Median value of outcome $y$.              \\
        \bottomrule
    \end{tabular}
    \vspace{0.5cm}
    
    \footnotesize{\textit{Notes:} Adapted from Table 1 in Wolfers and Zitzewitz (2004).}
\end{table}

Predictive markets have demonstrated their 
ability to aggregate dispersed information and 
improve forecasting accuracy in various domains. 
However, in the context of climate science, 
predictive markets remain underexplored. 
Traditional climate modeling faces key challenges: 

\begin{itemize}
    \item a lack of systematic assessment of forecasts, leading to information asymmetries, and
    \item the absence of direct financial incentives for accurate predictions.
\end{itemize}
This paper proposes climate predictive markets as 
a mechanism to enhance forecasting quality, 
incentivize expert participation, and 
improve societal preparedness for climate-related risks.
We examine the potential benefits of 
climate predictive markets for two key stakeholders:
\begin{itemize}
    \item Market Makers: What are the financial incentives for creating and maintaining such a market?
    \item Climate Experts (Smart Money): Can climate specialists leverage their knowledge profitably, and do they require the presence of less-informed participants to realize gains?
\end{itemize}
