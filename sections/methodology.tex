\section{Methodology}

\subsection{Market}


Consider a model that generates a forecast for a specific event, with result $A$ or $B$.
Let $s \in \{0, 1\}$ denote the outcome realization, where $s = 1$ if 
event $A$ occurs and $s = 0$ otherwise. Let $p_t$ denote the probability 
with which the event $s = 1$ is predicted to occur, according to the model 
forecast at data $t$. 

Now suppose that there exists a prediction market listing a contract 
that pays one dollar if event $A$ occurs and zero dollars otherwise. 
Let $q_t$ denote the price of this contract at date $t$. This price 
may differ from the model forecast $p_t$, in which case a trader who 
believes the model will recognize a profit opportunity.

To facilitate trading, we introduce an automated market maker who sets the price
$q_t$ of the contract. We employ the Logarithmic Market Scoring Rule (LMSR),
a market-making algorithm that dynamically adjusts prices based on trader demand 
while maintaining a bounded cost function. 
The LMSR mechanism ensures liquitdity at all times and updates the contract 
price to reflect the collective information of market participants.

The LMSR maintains a market consisting of a single market maker who sets 
the price of the event-contingent contract. The market maker employs 
a cost function $C(q_A, q_B)$ that determines the total cost 
required to purchase a given quantity of contracts. 
The LMSR cost function is given by:

\begin{equation}
    C(q_A, q_B) = b \ln (e^{q_A / b} + e^{q_B / b})
\end{equation}

where $q_A$ and $q_B$ are the number of outsanding contracts for outcomes 
$A$ and $B$ respectively, and $b$ is a liquidity parameter that determines  
the curvature of the cost function (ie. how quickly prices adjust).

The price of a contract for outcome $A$, denoted as $q_t$ is derived as:

\begin{equation}
    q_t = \frac{e^{q_A / b}}{e^{q_A / b} + e^{q_B / b}}
\end{equation}

Similarly, the price of a contract for outcome $B$ is given by:

\begin{equation}
    1 - q_t = \frac{e^{q_B / b}}{e^{q_A / b} + e^{q_B / b}}
\end{equation}

As traders buy and sell contracts, the outstanding quantities 
$q_A$ and $q_B$ are updated, and the price $q_t$ is recalculated.
The LMSR guarantees that the market price continuously 
reflects the aggregate information of traders by ensuring that the price 
update follows a smooth function of the quantity of contracts traded.

The total cost incurred by a trader purchasing $\Delta q_A$ additional 
contracts for outcome $A$ is given by:

\begin{equation}
    C(q_A + \Delta q_A, q_B) - C(q_A, q_B) = b \ln (1 + e^{\Delta q_A / b})
\end{equation}

This mechanism prevents arbitrage opportunities while ensuring that no individual 
trader can manipulate prices without incurring increasing costs. 

Under the LMSR mechanism, the contract price $q_t$ represents 
a market-implied probability of the event occuring. If market participants 
collectively believe the model's probability $p_t$ is incorrect, 
trading will drive the contract price toward the consensus belief. 
Consequently, the LMSR algorithm serves as a continuous 
price discovery mechanism, aligning $q_t$ with the collective
expectations of market participants.

\subsection{Participants}

Suppose that a virtual trader or bot that believes the model is active 
in the market, and enters period $y$ with a portfolio $(y_{t-1}, z_{t-1})$
inherited from the previous period. Here, $y_{t-1}$ is the amount of cash held 
at the start of period $t$, and $z_{t-1}$ is the number of contracts held. It 
is possible for $z_{t-1}$ to be nefative, in which case the trader has 
previously sold (rather than purchased) contracts on balance, and is 
therefore betting that event $A$ will not occur.

The trader must now decide whether to buy or sell contracts in period $t$,
having observed the model forecast $p_t$ and the market price $q_t$. 
Let $x_t$ denote the number of contracts trader in period $t$. 
If $x_t > 0$ then contracts are purchased, $z_t$ exceeds $z_{t-1}$ by this amount 
and $y_t$ is lower than $y_{t-1}$ by an amount equal to the cost 
$q_t x_t$ of the contracts bought. 
If $x_t < 0$, then contracts are sold, $z_t$ is lower than $z_{t-1}$ by this 
amount and $y_t$ is higher than $y_{t-1}$ by an amount equal to the 
revenue $q_t x_t$ from the sale of the contracts. That is, the portfolio 
passed on to the subsequent period is given by:

\begin{equation}
    \begin{aligned}
        y_t & = y_{t-1} - q_t x_t \\
        z_t & = z_{t-1} + x_t
    \end{aligned}
\end{equation}

But how is $x_t$ determined? When all contracts are resolved, 
the terminal cash value or wealth resulting from portfolio $(y_t, z_t)$ is:

\begin{equation}
    W = y_t + s z_t 
\end{equation}

That is, the trader is left with $y_t + z_t$ if $s = 1$ and $y_t$ otherwise.
We assume that the trader is risk-averse and maximizes the expected utility 
of terminal wealth, given by: 

\begin{equation}
    E(u) = p_t u (y_t + z_t) + (1 - p_t) u (y_t)
\end{equation}

where $u$ is a concave utility function. 
This objective function can be written in terms of the decision variable 
$x_t$, the interhited portfolio $(y_{t-1}, z_{t-1})$, the model forecast
$p_t$ and the market price $q_t$ by substituting the expressions for
$y_t$ and $z_t$ into the expression for $E(u)$:

\begin{equation}
    E(u) = p_t u (y_{t-1} - q_t x_t + x_t) + (1 - p_t) u (y_{t-1} - q_t x_t)
\end{equation}

This is the expected value of terminal wealth given current beliefs, 
conditional on the portfolio $(y_t, z_t)$, being held to expiration.
The trader knows that beliefs are likely to change, but cannot know the extent
or direction of these future shifts, and accordingly optimizes 
based on the model forecast and market price available at $t$. 

A trading bot programmed to execute transactions in accordance with 
the maximization of expected utility will trade whenever there is a change 
in the model forecast or the market price. 

Any such trades must be consistent with solvency even in the worst case scenario.
That is, the trading bot chooses $x_t$ to 
maximize the expected utility of terminal wealth, subject to the constraint

\begin{equation}
    \begin{aligned}
        y_t \geq 0 \\
        y_t + s z_t \geq 0
    \end{aligned}
\end{equation}

The first constraint states simply that the cash position at any date cannot become 
negative, and the second states that the trader must have enough cash to cover 
all obligations if contract holdings are negative (so the trader is betting against the event A)
and it turns out that event A occurs.

We look for itendifying hypothetical 
trades that would have been executed by the trader on each period,
starting with an initial cash position $y_0 = \$ 1000$
and an initial contract position $z_0 = 0$. 
The market has two contracts 
referencing the same event, one for each outcome.

In order to determine the sequence of trades $x_t$, 
we need to specify the utility function 
that represents the preferences of this virtual trader. 
A useful class of functions for which the degree of risk aversion 
can be tuned for experimental purposes is that 
exhibiting constant relative risk aversion (CRRA):

\begin{equation}
    \begin{aligned}
        u(w) = \frac{1}{1 - \rho} w^{1 - \rho} \quad \text{for} \quad \rho \geq 0, \rho \neq 1 \\
        u(w) = \ln w \quad \text{for} \quad \rho = 1
    \end{aligned}
\end{equation}

where $\rho$ is the coefficient of relative risk aversion. 
We assume the simple case of $\rho = 1$, corresponding to logarithmic utility.

\subsection{Event and Models}

We consider a simple time serie $X_t$
that evolves over time according to a stochastic process 
with a deterministic trend component. 
The event of intereset is whether this series exceeds a 
predefined threshold $\theta$ at the end of the period $T$. 
Formally, we define the event as:

\begin{equation}
    s = 1(X_T \geq \theta)
\end{equation}

where $1(\cdot)$ is the indicator function that takes the value 
1 if the condition is met and 0 otherwise. 

The probability of this event occuring at each time $t$ is denoted by $p_t$,
which represents the subjective probability assigned by 
the forecasting model. 

We consider two distinct types of markets participants,
each with a different approach to estimating $p_t$. 
These two participant represent different levels of 
sophistication in their forecasting methods.

\subsubsection{Naive Bayesian Trader}

The first type of trader follows a simple 
Bayesian updating rule, treating the most recently observed 
value $X_t$ as the best available information to 
update their beliefs about $s$. This trader assumes 
that $X_t$ follows a random walk with constant variance and updates 
their probability estimate as:

\begin{equation}
    p_t = P(X_T \geq \theta | X_t) \propto P(X_t \geq \theta | X_t, \sigma^2)
\end{equation}

where $\sigma^2$ represents the perceived variance of future increments.

Under the naive assumption that $X_t$ follwos a normal distribution 
centered at $X_t$, the Bayesian trader sets:

\begin{equation}
    p_t = 1 - \Phi \left( \frac{\theta - X_t}{\sigma \sqrt{T - t}} \right)
\end{equation}

where $\Phi(\cdot)$ is the standard normal cumulative distribution function.
This trader only reacts to the latest observed value.

\subsubsection{Statistical Traders (Sophisticated)}

The second type of participant represents an expert trader, 
who incorportes a simple statistical model to 
make a more informed forecast. Instead of relying solely on the latest 
value $X_t$, this trader fits a linear trend model 
to the observed series and projects forward to estimate $p_t$. 
Assuming a simple trend model:

\begin{equation}
    X_t = \alpha + \beta t + \epsilon_t
\end{equation}

where $\alpha$ and $\beta$ are the intercept and slope of the trend line,
and $\epsilon_t$ is a white noise error term.
The expert estimates $\alpha$ and $\beta$ using rolling regression 
over past observations and then predicts the final value:

\begin{equation}
    \hat{X}_T = \hat{\alpha} + \hat{\beta} T
\end{equation}

The expert then derives their probability estimates as:

\begin{equation}
    p_t = 1 - \Phi \left( \frac{\theta - \hat{X}_T}{\sigma \sqrt{T - t}} \right)
\end{equation}

Since this trader incoporates the estimated 
trend $\hat{\beta}$, their probability estimates may 
significantly differ from those of the Bayesian trader, 
especially in the presence of strong upward or downward trends in $X_t$.